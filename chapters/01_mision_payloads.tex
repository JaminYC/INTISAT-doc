
\chapter{Introduction}
\pagestyle{fancy}

\section{Mission Description}
INTISAT is a 2U CubeSat-class technology demonstration mission led by the Pontifical Catholic University of Peru (PUCP) through the Instituto de Radioastronomía (INRAS). The mission aims to validate the integration of a compact image-based microscopy payload for Earth orbit operation, developed by students and researchers as part of PUCP’s space education and technology development efforts.

INTISAT will demonstrate a fully integrated platform composed of an On-Board Computer (OBC), Electrical Power System (EPS), Telemetry, Tracking and Command (TTC) system, structural components, and a custom scientific payload. The system is enclosed within a 2U CubeSat mechanical envelope. The current implementation corresponds to an Engineering Model, serving as a foundation for future improvements and eventual flight qualification.

\insertfigure{IntiSat_Renderization.png}{0.25}{INTISAT 3D model view.}

\section{Mission Objectives}
The primary objectives of INTISAT are:
\begin{itemize}
    \item To validate the integration and operation of a miniaturized image sensor for microscopy in low Earth orbit (LEO).
    \item To demonstrate the functionality of student-developed subsystems working within a unified nanosatellite platform.
    \item To develop and verify a scalable and modular CubeSat architecture aligned with academic and educational standards.
    \item To serve as a hands-on training platform for students in satellite systems engineering, design, integration, and testing.
\end{itemize}

\section{Project Members}
INTISAT is an academic initiative driven by interdisciplinary teams at PUCP. The project structure includes:
\begin{itemize}
    \item \textbf{INRAS – PUCP}: Mission coordination, systems engineering, TTC module development, and platform integration.
    \item \textbf{Faculty of Science and Engineering – PUCP}: Development of software, electronics, and mechanical subsystems.
    \item \textbf{External partners (future phase)}: Collaborators for testing, launch coordination, or mission operations.
\end{itemize}

\section{Mission Patch}
\insertfigure{pucp_inras_logo.png}{0.25}{Official mission patch of INTISAT.}
