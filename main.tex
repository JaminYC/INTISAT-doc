
\documentclass[12pt]{report}

\usepackage{pucp_intisat}  % Estilo centralizado

\title{INTISAT – Detailed Design Report (CDR)}
\author{Instituto de Radioastronomía PUCP (INRAS)}
\date{\today}

\begin{document}

\begin{titlepage}
    \centering
    \vspace*{3cm}
    \includegraphics[width=0.4\textwidth]{figures/pucp_inras_logo.png}\par\vspace{1cm}
    {\scshape\Large PUCP – Instituto de Radioastronomía (INRAS) \par}
    \vspace{1.5cm}
    {\Huge\bfseries INTISAT Design and Mission Overview \par}
    \vspace{0.5cm}
    {\large\itshape Critical Design Review (CDR) – Engineering Model\par}
    \vspace{0.5cm}
    \rule{\textwidth}{0.4pt}
    \vspace{0.5cm}
    {\small Departamento de Ingeniería – Pontificia Universidad Católica del Perú \par}
    \vfill
    {\large November 2025\par}
\end{titlepage}



\clearpage
\listoffigures
\clearpage
\listoftables
\clearpage

\chapter*{Nomenclature}
\addcontentsline{toc}{chapter}{Nomenclature}
\begin{tabular}{ll}
OBC & On-Board Computer \\
EPS & Electrical Power System \\
TTC & Telemetry, Tracking and Command \\
CDR & Critical Design Review \\
PUCP & Pontificia Universidad Católica del Perú \\
INRAS & Instituto de Radioastronomía \\
\end{tabular}

\clearpage
\tableofcontents
\clearpage
%\pagestyle{fancy} % Reactivar después de portada
% Capítulos principales

\chapter{Introduction}
\pagestyle{fancy}

\section{Mission Description}
INTISAT is a 2U CubeSat-class technology demonstration mission led by the Pontifical Catholic University of Peru (PUCP) through the Instituto de Radioastronomía (INRAS). The mission aims to validate the integration of a compact image-based microscopy payload for Earth orbit operation, developed by students and researchers as part of PUCP’s space education and technology development efforts.

INTISAT will demonstrate a fully integrated platform composed of an On-Board Computer (OBC), Electrical Power System (EPS), Telemetry, Tracking and Command (TTC) system, structural components, and a custom scientific payload. The system is enclosed within a 2U CubeSat mechanical envelope. The current implementation corresponds to an Engineering Model, serving as a foundation for future improvements and eventual flight qualification.

\insertfigure{IntiSat_Renderization.png}{0.25}{INTISAT 3D model view.}

\section{Mission Objectives}
The primary objectives of INTISAT are:
\begin{itemize}
    \item To validate the integration and operation of a miniaturized image sensor for microscopy in low Earth orbit (LEO).
    \item To demonstrate the functionality of student-developed subsystems working within a unified nanosatellite platform.
    \item To develop and verify a scalable and modular CubeSat architecture aligned with academic and educational standards.
    \item To serve as a hands-on training platform for students in satellite systems engineering, design, integration, and testing.
\end{itemize}

\section{Project Members}
INTISAT is an academic initiative driven by interdisciplinary teams at PUCP. The project structure includes:
\begin{itemize}
    \item \textbf{INRAS – PUCP}: Mission coordination, systems engineering, TTC module development, and platform integration.
    \item \textbf{Faculty of Science and Engineering – PUCP}: Development of software, electronics, and mechanical subsystems.
    \item \textbf{External partners (future phase)}: Collaborators for testing, launch coordination, or mission operations.
\end{itemize}

\section{Mission Patch}
\insertfigure{pucp_inras_logo.png}{0.25}{Official mission patch of INTISAT.}

\chapter{Arquitectura del Sistema}
\section{Visión general del sistema}
Descripción del satélite completo.

\chapter{OBC – Computadora de a Bordo}
\section{Diseño de hardware y software}

\chapter{EPS – Sistema de Energía}
\section{Generación, almacenamiento y distribución}

\chapter{ADCS – Control de Actitud}
\section{Sensores, actuadores y control}

\chapter{TTC – Telecomunicaciones}
\section{Enlaces UHF/S-band y protocolos}

\chapter{Carga Útil}
\section{Diseño e integración del sistema óptico}

\chapter{Estructura y Mecánica}
\section{Diseño estructural y análisis}

\chapter{Control Térmico}
\section{Requerimientos y simulaciones térmicas}

\chapter{Segmento Tierra}
\section{Estación terrena, control y monitoreo}

\chapter{Integración, Verificación y Validación}
\section{Plan de pruebas y validación}

\chapter{Gestión del Proyecto}
\section{Cronograma, WBS y riesgos}


% Apéndices
\appendix
\chapter{Referencias}
\begin{itemize}
    \item ECSS-E-ST-10C – Requisitos de ingeniería de sistemas espaciales
    \item NASA Systems Engineering Handbook
\end{itemize}

\chapter{Anexos}
\section{Diagramas, tablas y presupuestos adicionales}


\end{document}
